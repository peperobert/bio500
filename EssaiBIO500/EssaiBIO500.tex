\documentclass[9pt,twocolumn,twoside,]{pnas-new}

% Use the lineno option to display guide line numbers if required.
% Note that the use of elements such as single-column equations
% may affect the guide line number alignment.


\usepackage[T1]{fontenc}
\usepackage[utf8]{inputenc}

% tightlist command for lists without linebreak
\providecommand{\tightlist}{%
  \setlength{\itemsep}{0pt}\setlength{\parskip}{0pt}}


% Pandoc citation processing
\newlength{\cslhangindent}
\setlength{\cslhangindent}{1.5em}
\newlength{\csllabelwidth}
\setlength{\csllabelwidth}{3em}
\newlength{\cslentryspacingunit} % times entry-spacing
\setlength{\cslentryspacingunit}{\parskip}
% for Pandoc 2.8 to 2.10.1
\newenvironment{cslreferences}%
  {}%
  {\par}
% For Pandoc 2.11+
\newenvironment{CSLReferences}[2] % #1 hanging-ident, #2 entry spacing
 {% don't indent paragraphs
  \setlength{\parindent}{0pt}
  % turn on hanging indent if param 1 is 1
  \ifodd #1
  \let\oldpar\par
  \def\par{\hangindent=\cslhangindent\oldpar}
  \fi
  % set entry spacing
  \setlength{\parskip}{#2\cslentryspacingunit}
 }%
 {}
\usepackage{calc}
\newcommand{\CSLBlock}[1]{#1\hfill\break}
\newcommand{\CSLLeftMargin}[1]{\parbox[t]{\csllabelwidth}{#1}}
\newcommand{\CSLRightInline}[1]{\parbox[t]{\linewidth - \csllabelwidth}{#1}\break}
\newcommand{\CSLIndent}[1]{\hspace{\cslhangindent}#1}


\templatetype{pnasresearcharticle}  % Choose template

\title{La reproductibilité scientifique: une arme précieuse contre la
desinformation}

\author[a]{Pénélope Robert}

  \affil[a]{Université de Sherbrooke, Départment de biologie, 2500
Boulevard de l'Université, Sherbrooke, Québec, G1V 0A9}


% Please give the surname of the lead author for the running footer
\leadauthor{}

% Please add here a significance statement to explain the relevance of your work
\significancestatement{}


\authorcontributions{}



\correspondingauthor{\textsuperscript{} }

% Keywords are not mandatory, but authors are strongly encouraged to provide them. If provided, please include two to five keywords, separated by the pipe symbol, e.g:


\begin{abstract}
En réponse à l'article ``Des études trop rapides en temps de COVID?'' de
l'Agence Science Presse, l'éditorial vous propose des piste de réflexion
en reliant la crise de reproductibilité, la crise sanitaire ainsi que la
crise climatique. La thèse défendue est que les facteurs qui limitent la
reproductibilité scientique contribu à la désinformation et les fausses
nouvelles (``fake news''). L'auteur discutent également des enjeux qui
entravent la reproductivité tel que les conflits d'intérêt pouvant être
relié à de la fraud ainsi que les solutions possibles comme
l'accessibilité des données.
\end{abstract}

\dates{This manuscript was compiled on \today}
\doi{\url{www.pnas.org/cgi/doi/10.1073/pnas.XXXXXXXXXX}}

\begin{document}

% Optional adjustment to line up main text (after abstract) of first page with line numbers, when using both lineno and twocolumn options.
% You should only change this length when you've finalised the article contents.
\verticaladjustment{-2pt}



\maketitle
\thispagestyle{firststyle}
\ifthenelse{\boolean{shortarticle}}{\ifthenelse{\boolean{singlecolumn}}{\abscontentformatted}{\abscontent}}{}

% If your first paragraph (i.e. with the \dropcap) contains a list environment (quote, quotation, theorem, definition, enumerate, itemize...), the line after the list may have some extra indentation. If this is the case, add \parshape=0 to the end of the list environment.

\acknow{}

Malgré toutes les terribles conséquences de la COVID-19, la pandémie a
tout de même mis en lumière plusieurs réalités. Elle nous a montré qu'il
y a effectivement une crise de reproductibilité dans le domaine
scientifique qu'il est possible d'accélérer les délais de publication et
donc de découvertes pour ainsi agir rapidement sur des fondements
scientifiques. En effet, les serveurs de prépublication ont comme
avantage de réduire les délais de publication. Par contre, il y a un
revers de la médaille à cette rapidité et à son accessibilité (1): la
fiabilité des études est questionnable, car ce sont des recherches qui
n'ont pas été révisées. En théorie, cette rapidité et accessibilité est
très utile puisqu'elles permettent d'augmenter la révision par les pairs
et donc la rapidité du processus. Le problème rencontré durant la
pandémie est que les journalistes non aguerri et mal informé ont tiré
avantage de ces plateformes et ont publié des études non validées (1).
C'est justement une autre leçon que la pandémie nous a révélée, les
journalistes et éditeurs se doivent d'être plus rigoureux. Cela dit,
c'est exactement de cette façon-là que la crédibilité scientifique prend
un coup et laisse place à une panoplie de fausses études et à la
désinformation. Le seul outil que nous avons pour réacquérir notre
crédibilité est la reproductibilité.

\hypertarget{facteurs-qui-limitent-la-reproductibilituxe9}{%
\section{Facteurs qui limitent la
reproductibilité}\label{facteurs-qui-limitent-la-reproductibilituxe9}}

Lors un sondage émis par le journal \emph{Nature} sur la
reproductibilité les répondants ont mis en lumière les grandes lignes du
problème. Les scientifiques relatent que dans les 3 grands facteurs qui
affectent le plus la reproductibilité sont la sélectivité de
l'information, la pression de publier ainsi que le peu de puissance
statistique (2). Les deux premiers plus grands facteurs sont interreliés
puisque c'est la pression de publier qui pousse des chercheurs à
consciemment rapporter partiellement des informations ou des données
afin de faire parler les résultats comme ils le veulent. D'autres
facteurs peuvent également être le manque de mentorat, une mauvaise
méthode expérimentale et la fraude (qui selon moi est très proche de la
sélectivité de l'information) (2).

\hypertarget{conflits-dintuxe9ruxeats-et-fraude-scientifique}{%
\section{Conflits d'intérêts et fraude
scientifique}\label{conflits-dintuxe9ruxeats-et-fraude-scientifique}}

Le même sondage du journal \emph{Nature} a demandé quelles approches
devraient être améliorées. De meilleures analyses statistiques, de
meilleures méthodes et davantage de mentorat sont les répondes les plus
populaires. C'est vrai, mais il n'y a pas de pistes de solution pour le
plus gros problème: la pression de découvrir ce qu'on recherche et la
pression de publication. Selon moi, la partie prenante la plus
importante qui contribue à ce fléau est les sociétés de financement.
Évidemment, c'est très dispendieux une recherche. Le problème est que
les sociétés de financement, autant publiques que privées, fournissent
de l'argent aux études qui ont le potentiel ``prouver'' ce qui les
avantages. Il y a ici un gros conflit d'intérêts dans le processus.
Ironiquement, les études qui semblent être le moins reproductibles ont
moins de chance d'être publiées dans de grands journaux prestigieux.
Évidemment, les sociétés de financement publique joue un rôle de bon
samaritain puissent qu'elles poussent, théoriquement, pour le bien
commun. Les compagnies privées par contre peuvent en faire à leur tête.
Le meilleur exemple est celui de la crise climatique et de la compagnie
pétrolière Exxon. Depuis au moins 1977 qu'Exxon sait scientifiquement
que le CO2 produit par la combustion du pétrole contribuent au
réchauffement climatique (3). C'est 11 ans avant même que la NASA lance
l'alerte. Exxon a nié que leur conclusion n'était pas fiable. Depuis,
ils sont devenus des leaders dans la désinformation scientifique autour
de la crise climatique. C'est l'enquête sur Exxon par InsideClimate News
qui a mis au grand jour des dizaines et des dizaines de documents qui
relataient leur plan d'action. La citation suivante a été trouvée
(traduit de l'anglais) : ``La victoire sera atteinte lorsque le citoyen
moyen sera incertain des changements climatiques'' (3). Selon
Greenpeace, Exxon a financiers pour plus de 30 millions de dollars US
sur des recherches et des campagnes de publicité climatosceptiques (3).
Le lobbying du pétrole est un très bel exemple de fraude scientifique
qui perdure encore aujourd'hui.

\hypertarget{laccessibilituxe9-uxe0-linformation-brute}{%
\section{L'accessibilité à l'information
brute}\label{laccessibilituxe9-uxe0-linformation-brute}}

Une des façons d'éviter le fiasco d'Exxon et d'avoir des études fiables
c'est l'accessibilité des données. C'est un aspect fondamental et
indispensable à la reproductibilité qu'il ne faut pas sous-estimée.
C'est une action simple et facile puisque la technologie nous le permet
(4). Cette mesure permet également l'avancement de la science, sans
avoir besoin d'un gros budget. Effectivement, cela permet que d'autres
scientifiques utilisent les données soit pour répliquer l'expérience
avec d'autres analyses ou soit pour effectuer une méta-analyse (5). Dans
le même ordre d'idée, de nombreuses collaborations peuvent naitre de ce
partage d'informations et de piste de réflexion (4).

\hypertarget{de-lurgence-sanitaire-uxe0-lurgence-climatique}{%
\section{De l'urgence sanitaire à l'urgence
climatique}\label{de-lurgence-sanitaire-uxe0-lurgence-climatique}}

Nous avons donc vu jusqu'à présent qu'il est possible de faire accélérer
la science et être reproductibilité du même coup. L'accessibilité des
données et les serveurs de prépublication en sont des exemples.
Maintenant que nous savons que c'est possible : pourquoi n'agissons-nous
toujours pas sur la crise climatique? Pourquoi n'accélérons pas les
données et les études sur le sujet afin d'enterrer ``l'incertitude du
citoyen moyen?'' L'urgence climatique n'est-elle pas aussi si ce n'est
pas plus importante que l'urgence sanitaire? Sans surprise, c'est une
question d'argent. Le Canada a subventionné pour près de 330 millions de
dollars pour la recherche sur la COVID-19 (6). En 2019, c'est 175
millions de dollars qui sont débloqués pour la conservation (7). Rien du
tout depuis, mais le Ministre de l'Environnement, M. Guilbeault, accepte
le projet d'extraction pétrolier de la Bay du Nord deux jours après le
rapport exubérant du GIEC. Utilisons donc notre arme la plus précieuse,
la reproductibilité scientifique, afin de remettre les pendules à
l'heure et éviter d'autres catastrophes planétaires.

\pnasbreak

\hypertarget{refs}{}
\begin{CSLReferences}{0}{0}
\leavevmode\vadjust pre{\hypertarget{ref-radiocan}{}}%
\CSLLeftMargin{1. }
\CSLRightInline{Bilodeau M (2021) Des études trop rapide en temps de
COVID? \emph{Agence Science Presse}. Available at:
\url{https://www.scientifique-en-chef.gouv.qc.ca/impacts/ddr-des-etudes-trop-rapides-en-temps-de-covid/}.}

\leavevmode\vadjust pre{\hypertarget{ref-Baker2016}{}}%
\CSLLeftMargin{2. }
\CSLRightInline{Baker M (2016) 1,500 scientists lift the lid on
reproducibility. \emph{Nature} 533(7604):452--454.}

\leavevmode\vadjust pre{\hypertarget{ref-exxon}{}}%
\CSLLeftMargin{3. }
\CSLRightInline{Hall S (2015) Exxon knew about climate change almost 40
years ago. \emph{Scientific American}. Available at:
\url{https://cihr-irsc.gc.ca/f/52447.html}.}

\leavevmode\vadjust pre{\hypertarget{ref-Poisot2013}{}}%
\CSLLeftMargin{4. }
\CSLRightInline{Poisot T, Mounce R, Gravel D (2013) Moving toward a
sustainable ecological science: Don't let data go to waste! \emph{Ideas
in Ecology and Evolution} 6(2).
doi:\href{https://doi.org/10.4033/iee.2013.6b.14.f}{10.4033/iee.2013.6b.14.f}.}

\leavevmode\vadjust pre{\hypertarget{ref-mesirov}{}}%
\CSLLeftMargin{5. }
\CSLRightInline{Mesirov JP (2010) Accessible reproducible research.
\emph{Science} 327(5964):415--416.}

\leavevmode\vadjust pre{\hypertarget{ref-covidcash}{}}%
\CSLLeftMargin{6. }
\CSLRightInline{Canada I de recherche en santé (2022) Investissements
des IRSC liés à la COVID-19 : Les chiffres. \emph{Agence Science
Presse}. Available at: \url{https://cihr-irsc.gc.ca/f/52447.html}.}

\leavevmode\vadjust pre{\hypertarget{ref-ecocash}{}}%
\CSLLeftMargin{7. }
\CSLRightInline{Canada E et changement climatique (2019) Les
investissements fédéraux de 175 millions de dollars dans la nature
donnent le coup d'envoi à des projets de conservation dans chaque
province et territoire. Available at:
\url{https://www.canada.ca/fr/environnement-changement-climatique/nouvelles/2019/08/les-investissements-federaux-de-175-millions-de-dollars-dans-la-nature-donnent-le-coup-denvoi-a-des-projets-de-conservation-dans-chaque-province-et.html}.}

\end{CSLReferences}



% Bibliography
% \bibliography{pnas-sample}

\end{document}
