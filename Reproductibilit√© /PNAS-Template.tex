\documentclass[9pt,twocolumn,twoside,]{pnas-new}

% Use the lineno option to display guide line numbers if required.
% Note that the use of elements such as single-column equations
% may affect the guide line number alignment.


\usepackage[T1]{fontenc}
\usepackage[utf8]{inputenc}

% tightlist command for lists without linebreak
\providecommand{\tightlist}{%
  \setlength{\itemsep}{0pt}\setlength{\parskip}{0pt}}




\templatetype{pnasresearcharticle}  % Choose template

\title{Les collaborations entre les étudiants en BIO500 et leur
ressemblance aux réseaux écologiques}

\author[a]{Pénélope Robert}
\author[a]{Édouard Nadon-Beaumier}
\author[a]{Alexis Matte}
\author[a]{Nadia Tardy}

  \affil[a]{Université de Sherbrooke, Départment de biologie, 2500
Boulevard de l'Université, Sherbrooke, Québec, G1V 0A9}


% Please give the surname of the lead author for the running footer
\leadauthor{}

% Please add here a significance statement to explain the relevance of your work
\significancestatement{}


\authorcontributions{}



\correspondingauthor{\textsuperscript{} }

% Keywords are not mandatory, but authors are strongly encouraged to provide them. If provided, please include two to five keywords, separated by the pipe symbol, e.g:


\begin{abstract}
Nous avons cherché à vérifier si le nombre de collaborations par
étudiant de BIO500 est semblable ou bien s'il y a des ``étudiants
clés'\,'. Pour ce faire, nous avons effectué une analyse visuelle et
graphique, puis en deuxième lieu une analyse statistique de notre
distribution. La majorité des étudiants du cours BIO500 ont eu entre 10
et 25 collaborations et plus précisément entre 10 et 15. Seulement 2
étudiants ont entre 25 et 35 collaborations, soit Laura Béland, avec 34
collaborations et Édouard Nadon-Beaumier avec 29 collaborations. Les
données des collaborations respectent une distribution normale, mais
avec un p-value de 0.08508, très proche du seuil.
\end{abstract}

\dates{This manuscript was compiled on \today}
\doi{\url{www.pnas.org/cgi/doi/10.1073/pnas.XXXXXXXXXX}}

\begin{document}

% Optional adjustment to line up main text (after abstract) of first page with line numbers, when using both lineno and twocolumn options.
% You should only change this length when you've finalised the article contents.
\verticaladjustment{-2pt}



\maketitle
\thispagestyle{firststyle}
\ifthenelse{\boolean{shortarticle}}{\ifthenelse{\boolean{singlecolumn}}{\abscontentformatted}{\abscontent}}{}

% If your first paragraph (i.e. with the \dropcap) contains a list environment (quote, quotation, theorem, definition, enumerate, itemize...), the line after the list may have some extra indentation. If this is the case, add \parshape=0 to the end of the list environment.

\acknow{}

\hypertarget{introduction}{%
\section{Introduction}\label{introduction}}

La notion de réseau écologique d'un écosystème est souvent déformée et
confondue avec celle de réseau trophique. Nous avons régulièrement
tendance à imager les interactions entre les différentes espèces avec
une belle pyramide qui retrace les réseaux trophiques et la chaîne
alimentaire en plaçant le plus important prédateur au sommet du prisme.
En réalité, un réseau écologique inclut toutes les multitudes
d'interactions possibles entre les espèces comme le mutualisme, la
compétition, le commensalisme et le parasitisme
(\textbf{10.1371/journal.pcbi.1002928?}). Évidemment, les réseaux
trophiques ont une grande influence sur le réseau écologique, mais ce
dernier ne devrait pas être représenté par une pyramide. Il devrait
plutôt être décrit comme une toile (``web'') avec les espèces clés de
l'écosystème au centre, c'est-à-dire les espèces qui ont le plus grand
nombre de connexions interspécifiques. On retrouve également dans la
littérature les propriété de petit monde que l'on peut découvrir dans
les réseaux (\textbf{watts\_collective\_1998?}). C'est dans ce contexte
que nous avons voulu comparer le réseau de collaboration entre les
étudiants en écologie et le réseau écologique. Afin d'approfondir notre
réflexion, nous avons voulu vérifier si le nombre de collaborations par
étudiant de BIO500 est semblable ou bien s'il y a des ``étudiants
clés,'' c'est-à-dire qui possèdent le plus de collaborations différentes
au même titre qu'une espèce clé.

\hypertarget{muxe9thode}{%
\section{Méthode}\label{muxe9thode}}

Avant même de se pencher sur nos questions, il a fallu que chacun des
étudiants de BIO500 recense l'entièreté des étudiants avec qui il ou
elle a collaboré dans un travail d'équipe pendant son parcours
académique en écologie à l'Université de Sherbrooke. Une fois les
données recueillies et traitées, nous avions toutes les informations en
main pour commencer notre analyse. Afin de comparer en général le réseau
étudiant et les réseaux écologiques, nous avons généré deux figures: le
réseau de liens des étudiants de BIO500 et leurs collaborations ainsi
qu'un diagramme à bandes du nombre de collaborations pour chacun des
étudiants de BIO500.

Ensuite, pour répondre à notre question, nous avons tout d'abord émis
une hypothèse. Selon nous, le nombre de collaborations par personne
devrait être environ semblable d'un étudiant à l'autre puisque le
parcours académique et les cours sont relativement les mêmes. Nous nous
attendons donc à ce que le nombre de collaborations par étudiant suive
une distribution normale. En fait, c'est la méthode utilisée par les
écologistes pour savoir si le réseau qu'ils étudient a un nombre
``normal'' de liens (\textbf{MACDONALD2020100079?}). Si ce n'est pas le
cas, nous nous pencherons sur l'hypothèse des ``étudiants clés.''
L'objectif sera de les identifier et de déterminer la cause de leur haut
nombre de collaborateurs.

En premier lieu, nous avons effectué une analyse visuelle et graphique,
puis en deuxième lieu une analyse statistique de notre distribution. Les
deux analyses, comme les traitements de données ci-haut, ont été
effectuée sur le logiciel R version 4.0.3. L'analyse visuelle est
représentée par un histogramme, voici 4 résultats possibles ainsi que
leur signification (\textbf{MACDONALD2020100079?}):

\textbf{Une courbe symétrique et mince}

C'est à quoi nous nous attendions. La variance entre le nombre de
collaborations entre les élèves de BIO500 est faible et très concentrée
autour de la moyenne. Les étudiants ont donc eu un parcours académique
semblable.

\textbf{Une courbe symétrique et large}

Cette courbe corrobore également notre hypothèse: une distribution
normale. Par contre, la variation entre les élèves est plus grande
qu'attendu. Il y a une petite divergence entre les parcours, mais cela
ne semble pas débalancer les données.

\textbf{Une courbe asymétrique à gauche}

Ce genre de distribution non-normale indique que la majorité des élèves
se retrouvent malgré tout proche de la moyenne. Cependant, un certain
nombre non-négligeable d'élèves ont plus de collaborations et aucun ou
très peu d'élèves ont peu collaboré. Cette courbe nous indique qu'il y a
plusieurs ``étudiants clés'' plus ou moins importants.

\textbf{Une courbe symétrique avec des données aberrantes vers la
droite}

On observe une distribution normale, mais avec ce qui semble être des
données aberrantes. Cela représente précisément l'autre hypothèse
``d'étudiants clés'' peu nombreux et très importants. Leur divergence
distinctive pourrait signifier une grande différence de leur parcours.

L'analyse statistique est effectuée par le test de Shapiro-Wilk dans
lequel les hypothèses sont les suivantes:

Hypothèse nulle (H0): les données ont une distribution normale Hypothèse
alternative (H1): les données n'ont pas une distribution normale

La commande \texttt{shapiro.test()} est facile d'utilisation et sa
sortie nous informe sur la valeur de W et le p-value. C'est seulement le
p-value qui nous intéresse. Si le p-value est plus petit que 0.05, on
rejette l'hypothèse nulle: les données sont non-normales. Si p-value est
plus grand que 0,05, on ne rejette pas l'hypothèse nulle: les données
ont donc une distribution normale.

\hypertarget{analyse-et-discussion}{%
\section{Analyse et discussion}\label{analyse-et-discussion}}

\begin{figure}
\centering
\includegraphics[width=0.5\textwidth,height=0.4\textheight]{"../donnees_BIO500/igraph_BIO500.png"}
\caption{Réseau de collabarations entre les élèves de BIO500.
\label{fig:plot1}}
\end{figure}

La Figure \ref{fig:plot1} nous permet d'observer la complexité du réseau
de collaboration et la dynamique entre les étudiants.

\begin{figure}
\centering
\includegraphics[width=0.5\textwidth,height=0.4\textheight]{"../donnees_BIO500/diagram_collab2.pdf"}
\caption{\label{fig:plot2}}
\end{figure}

Dans la Figure \ref{fig:plot2}, on remarque que la moyenne semble être
autour de 15 collaborateurs. Il semble également y avoir deux étudiants
qui se démarquent du lot. Nous pouvons les considérer comme des
``étudiants clés'' potentiels.

\begin{figure}
\centering
\includegraphics[width=0.5\textwidth,height=0.4\textheight]{"../donnees_BIO500/histoBIO500.pdf"}
\caption{\label{fig:plot3}}
\end{figure}

L'histogramme de la fréquence du nombre de collaborations des étudiants
de BIO500, la Figure \ref{fig:plot3}, semble plutôt balancé. La majorité
des étudiants de BIO500 ont eu entre 10 et 15 collaborations comme
observé dans la figure 2. Par contre, l'allure de la courbe rend notre
analyse mitigée. Notre courbe n'est pas tout à fait symétrique, mais son
asymétrie vers la gauche n'est pas totalement importante.

L'analyse statistique peut nous en dire plus. Le test de Shapiro-Wilk
nous relève un p-value de 0.08508. Puisque qu'il est au-dessus de 0.5
nous ne pouvons donc pas rejeter l'hypothèse nulle: la distribution est
donc normale. L'hypothèse de notre question nous est donc confirmée: le
nombre de collaborations entre les élèves de BIO500 est relativement
semblable et suit une distribution normale. Par contre, notre p-value
est très proche du seuil d'acceptation. Nous voulons donc pousser notre
réflexion plus loin. Pourquoi est-ce que notre réseau est plus ou moins
normal selon les standards de nombre de liens entre les espèces? Selon
Arthur et ses collaborateurs (\textbf{MACDONALD2020100079?}), même en
écologie vérifier la normalité du nombre de liens dans un réseau n'est
pas tout à fait adéquat. Effectivement, il faut savoir que cette méthode
a été développée à l'aide de simulations de distribution de résaux
``parfaits'' et sans biais. Alors, cette analyse suppose que les
connexions ont toutes des propriétés fixes dans le réseau. En d'autres
mots, la méthode ne prend pas en compte la stochasticité d'un
écosystème, dans notre cas, la stochasticité des parcours académiques.
Nous avons assumé, tout comme la méthode que nous avons utilisé, que
chaque étudiant de BIO500 a eu le même parcours. Ce n'est effectivement
pas le cas. Le cours de BIO500 est le cours obligatoire de 6ème session
(3ème année) des étudiants en écologie. Puisque nous sommes en hiver
2022, la majorité des étudiants devraient être rentréa à l'université en
automne 2019. Effectivement, 39 étudiants sur 47 sont de la même
cohorte. Le reste des étudiants ont des parcours divergents, avec plus
ou moins de cours de terminés alors inévitablement un nombre de
collaborations différent, c'est-à-dire de la stochasticité.

\textbf{``Étudiants-clés''}

Malgré avoir prouvé la normalité de notre distribution, notre curiosité
nous a poussé à identifier les deux ``étudiants-clé'' potentiels: Laura
Béland et Édouard Nadon-Beaumier avec 34 et 29 collaborations. On peut
considérer ces personnes les plus connectées dans le réseau: ils ont le
plus de liens différents avec la totalité du réseau. Dans un réseau
écologique, ils seraient les espèces les plus importantes pour garder
l'intégrité du système puisqu'ils ont une large étendue de connexions.



% Bibliography
% \bibliography{pnas-sample}

\end{document}
